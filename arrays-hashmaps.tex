\documentclass[12pt, letterpaper]{report}
\usepackage[a4paper,
            bindingoffset=0.2in,
            left=1in,
            right=1in,
            top=1in,
            bottom=1in,
            footskip=.25in]{geometry}
\usepackage{graphicx} % Required for inserting images
\usepackage{listings}
\usepackage{xcolor}
\usepackage{color}
\usepackage[T1]{fontenc}
\usepackage{inconsolata}
\usepackage{fancyhdr}
\usepackage{extramarks}
\usepackage{titlesec}


% Dracula colors, per dracula
\definecolor{draculabg}      {RGB} {40,   42,   54}
\definecolor{draculacl}      {RGB} {68,   71,   90}
\definecolor{draculafg}      {RGB} {248,  248,  242}
\definecolor{draculacomment} {RGB} {98,   114,  164}
\definecolor{draculacyan}    {RGB} {139,  233,  253}
\definecolor{draculagreen}   {RGB} {80,   250,  123}
\definecolor{draculaorange}  {RGB} {255,  184,  108}
\definecolor{draculapink}    {RGB} {255,  121,  198}
\definecolor{draculapurple}  {RGB} {189,  147,  249}
\definecolor{draculared}     {RGB} {255,  85,   85}
\definecolor{draculayellow}  {RGB} {241,  250,  140}
\pagecolor{draculabg}
\color{draculafg}


% Header and footer syles
\newcommand{\sectionTop}[1]{
   \heading{#1}
   \sectionContinue{#1}

   \renewcommand{\sectionContinue}[1]{
        \afterpage{
           \heading{#1 Continued}
           \sectionContinue{#1}
        }
   }
}

\fancypagestyle{plain}{%
  \fancyhf{}%
  \renewcommand{\headrulewidth}{0.4pt}%
  \renewcommand{\footrulewidth}{0.4pt}%
  \fancyhead[R]{\itshape\nouppercase{\rightmark}}%
  \fancyfoot[C]{\thepage}
}
\fancypagestyle{general}{%
  \fancyhf{}%
  \renewcommand{\headrulewidth}{0.4pt}%
  \renewcommand{\footrulewidth}{0.4pt}%
  \fancyhead[R]{\itshape\nouppercase{\rightmark}}%
  \fancyfoot[C]{\thepage}
}
\pagestyle{general}

% Remove space above title
\usepackage{titling}
%\setlength{\droptitle}{-5em}
% \setlength{\voffset}{-5em}


% Decrease space above chapters
\titleformat{\chapter}[block]
  {\normalfont\huge\bfseries}{\chaptertitlename\ \thechapter{: }}{20pt}{\Huge}
\titlespacing*{\chapter}{0pt}{-40pt}{40pt}

% Doc Data
\title{\textbf{Arrays (and Hashmaps)}}
\author{Ya boi, BWalsh }

% Set default paragraph indent to 0
\setlength{\parindent}{0pt}

% Code styling
%\definecolor{bluekeywords}{rgb}{0.13,0.13,1}
%\definecolor{greencomments}{rgb}{0,0.5,0}
%\definecolor{redstrings}{rgb}{0.9,0,0}
\lstset{language={Python},
  showspaces=false,
  showtabs=false,
  breaklines=true,
  showstringspaces=false,
  breakatwhitespace=true,
  escapeinside={(*@}{@*)},
  commentstyle=\color{draculagreen},
  stringstyle=\color{draculared},
	identifierstyle=\color{white},
	numberstyle=\color{draculacomment},
  basicstyle=\ttfamily,
  literate = *{\ \ }{\ }1,
	emph={string,int,char,double,float,unsigned,void,bool},
    emphstyle={\color{draculacyan}},
		otherkeywords={>,<,.,;,-,!,=,~,&,|},
    keywordstyle=\color{draculapink},
		alsodigit={-},
}

% Also code styling, but highlights digits
\usepackage{etoolbox}
\newtoggle{InString}{}% Keep track of if we are within a string
\togglefalse{InString}% Assume not initally in string

\newcommand*{\ColorIfNotInString}[1]{\iftoggle{InString}{#1}{\color{draculaorange}#1}}%
\newcommand*{\ProcessQuote}[1]{#1\iftoggle{InString}{\global\togglefalse{InString}}{\global\toggletrue{InString}}}%
\lstset{literate=%
    {0}{{{\ColorIfNotInString{0}}}}1
    {1}{{{\ColorIfNotInString{1}}}}1
    {2}{{{\ColorIfNotInString{2}}}}1
    {3}{{{\ColorIfNotInString{3}}}}1
    {4}{{{\ColorIfNotInString{4}}}}1
    {5}{{{\ColorIfNotInString{5}}}}1
    {6}{{{\ColorIfNotInString{6}}}}1
    {7}{{{\ColorIfNotInString{7}}}}1
    {8}{{{\ColorIfNotInString{8}}}}1
    {9}{{{\ColorIfNotInString{9}}}}1
}

\begin{document}

\maketitle

\tableofcontents

\chapter{Overview}

\section{Arrays}
\subsection{Big-O Notation}
\begin{itemize}
    \item \textbf{Space}: \(O(n)\)
    \item \textbf{Search}: \(O(n)\)
    \item \textbf{Access}: \(O(1)\)
    \item \textbf{Insertion}: \(O(n)\)
    \item \textbf{Deletion}: \(O(n)\)
\end{itemize}

\subsection{Creation}

\begin{lstlisting}[language=Python,caption={Arrays in Python}]
    # Create empty list
    my_nums = []

    # Create list with values
    my_nums = [5,6,7]

    # Create list from string
    chars = list('Hello')
    # chars => ['h', 'e', 'l', 'l', 'o']

    # List from set
    unique_set = set(1,2,3)
    unique_list = list(unique_set)
    # unique_list => [1, 2, 3]

    // Size of unique_list
    arrLen = len(unique_list)
\end{lstlisting}

\pagebreak

\begin{lstlisting}[language=C,caption={Arrays in C}]
    int main() {
        // Create array with fixed size
        int array[5];
        array = {0, 1, 2, 3, 4};

        // Or in one fell swoop
        int array2[5] = {0, 1, 2, 3, 4};

        // With chars
        char myString = "Hello World";

        // Array with 5 items with malloc()
        int *myInts = (int) malloc(sizeof(int) * 5);

        // Size of myInts
        int arrLen = sizeof(myInts) / sizeof(myInts[0]);
    }
    
\end{lstlisting}

\vspace{\baselineskip} % Add a blank line or adjust the space as needed


\begin{lstlisting}[language=C++, caption={Arrays in C++}]
    #include <vector>
    #include <string>
    using namespace std;
    
    int main() {
        // Create a vector (a list) of strings
        vector<string> myStrings{ "hello", "world" };

        // Create a vector and add items after
        vector<int> myInts;
        myInts.push_back(7);
        myInts.push_back(2);

        // Size of myInts
        int arrLen = myInts.size();

        return 0;
    }
    
\end{lstlisting}

\pagebreak
\section{Hashmaps}
\subsection{Big-O Notation}

\begin{itemize}
    \item \textbf{Space}: \(O(n)\)
    \item \textbf{Get (Access)}: \(O(1)\)
    \item \textbf{Contains Key (Access)}: \(O(1)\)
    \item \textbf{Insertion}: \(O(1)\)
    \item \textbf{Delete Key}: \(O(1)\)
\end{itemize}

\subsection{Creation}

\begin{lstlisting}[language=Python,caption={Dicts (maps) in Python}]
    # Create empty dict
    fruits = {}

    # Add item to dict
    fruits['apple'] = 7

    # Create dict with items in it
    fruits = {
        'apple': 10,
        'banana': 2
    }

    # Create dict with 'dict' builtin func
    cat_names = dict(one='Ham', two='Beanbag')
    print(cat_names['two']) # => Beanbag

    # You can set a default value for keys 
    # using a default dict so you don't
    # get an error when trying to access
    # an undefined key. 
    #
    # This is very helpful for
    # frequency lists.
    from collections import defaultdict

    values = defaultdict(int)
    values['a'] = 10

    print(values['a']) # => 10
    print(values['b']) # => 0
\end{lstlisting}

\pagebreak

\begin{lstlisting}[language=C++,caption={Maps in C++}]
    #include <map>
    #include <string>
    #include <iostream>
    using namespace std;

    int main() {

        map<string, int> fruits;
        fruits.insert(pair<string,int>("Apples", 10));
        fruits.insert(pair<string,int>("Bananas", 2));

        // Fetch specific value
        map<string,int>::iterator it;
        it = fruits.find("Apples");

        cout << it->first << ": " << it->second;
        // => Apples: 10

        return 0;
    }
\end{lstlisting}

\pagebreak
\chapter{Algorithms}
\section{Binary Search}

Binary Search is a popular algorithm often used to find a target in an already-sorted list. The methodology goes as such:

\begin{itemize}
    \item Create two variables that will act as a \textbf{left pointer}  and a \textbf{right pointer}, with \lstinline{left} initialized to 0 to represent the beginning of the array and \lstinline{right} initialized to the last index of the array (size - 1)
    \item Start a while-loop for condition \lstinline{left <= right}
    \item{Calculate the middle index between \lstinline{left} and \lstinline{right}: 
        \newline 
        \lstinline{middle = left + (right - left) / 2}
    }
    \item Check if we found target, if so return
    \item{Check if current middle item is smaller than target. If so, discard the smaller (left-hand) half, and set our left pointer to 1 after the middle: \newline
        \lstinline{left = middle + 1}
    }
    \item{Otherwise, our current item is larger than the target. So we discard the larger half, and set our right pointer to 1 less than the middle: \newline
        \lstinline{right = middle - 1}
    }
\end{itemize}

The time complexity is $O(log(n))$. This is because at every iteration of the loop, we are halving what part of the array we're looking at. 

\begin{lstlisting}[language=Python,caption={Binary Search In Python}]
    def binary_search(items, target):
        left = 0
        right = len(items) - 1
    
        while left <= right:
            # // in Python rounds to the nearest integer
            mid = (left + right) // 2
    
            if items[mid] == target:
                return mid
            elsif items[mid] < target:
                left = mid + 1
            else items[mid] > target:
                right = mid - 1
                
        return -1

    binary_search([2,3,4,5,6,7,8,9], 3)
\end{lstlisting}

\begin{lstlisting}[language=C++,caption={Binary Search In C++}]
    int binary_search(vector<int>& items, int target) {
        int left = 0;
        int right = items.size() - 1;
    
        while (left <= right):
            int mid = left + (right - left) / 2;
    
            if(items[mid] == target) {
                return mid;
            } else if (items[mid] < target) {
                left = mid + 1;
            } else (items[mid] > target) {
                right = mid - 1;
            }
                
        return -1;
    }
\end{lstlisting}

\pagebreak
\section{Sliding Window}

Sliding Window is a technique used to search or traverse an array while looking at a subsection of the array (a "window") at a time. The goal of sliding window algorithms is to minimize nested loops and reduce their time complexities. These algorithms are used often for the following types of problems:

\begin{itemize}
    \item Minimum / Maximum Sum Array
    \item Longest Sequence / Substring
    \item K Closest Elements
\end{itemize}


Assume we have a problem where we want to find the largest subarray in a given array:
\begin{lstlisting}[language=Python,caption={Sliding Window In Python}]
    # k is the target size of the subarray
    def maximum_subarray(nums, k):
        # Set the current maximum to lowest number
        max_sum = float('-inf')

        # Keep track of local sums
        current_sum = 0

        # Calculate the initial max sum
        # in the first window (first k items)
        for i in range(k):
            current_sum += nums[i]

        # Loop through the rest of array
        for i in range(k, len(nums)):
            # Update the largest subarray that we've seen
            max_sum = max(max_sum, current_sum)

            # Add the current num to the current_sum
            # Subtracting the first item from the last window
            current_sum += nums[i] - nums[i - k]

        return max_sum
        

    maximum_subarray([5,8,3,6,9,1,0,9], 3)


    """
    So for the above example, we effectively end up seeing:

    Step 1:
    Our window is [5,8,3], the first k elements.
    The pointer is at index 3 (value of k), but we only look
    at the values before that (exclusive).
    
    [5, 8, 3, 6, 9, 1, 0, 9]
     ^        ^
     |        |
   left     right

    Subarray (window) sum: 5 + 8 + 3 = 16

    Step 2:
    [5, 8, 3, 6, 9, 1, 0, 9]
        ^        ^
        |        |
      left     right
  """
\end{lstlisting}

\pagebreak
\begin{lstlisting}[language=C,caption={Sliding Window In C}]
    #include <stdio.h>

    int maximumSubarray(int* nums, int numsSize, int target);
    int max(int, int);
    
    int main() {
      int nums[8] = {5, 8, 3, 6, 9, 1, 0, 9};
      int numsSize = sizeof(nums) / sizeof(int);
      int maxSum = maximumSubarray(nums, numsSize, 3);
      printf("Maximum sub in subarray: %d", maxSum);
    }
    
    int maximumSubarray(int* nums, int numsSize, int k) {
      int maxSum = -__INT_MAX__;
      int currentSum = 0;
    
      for (int i = 0; i < k; i++) {
        currentSum += nums[i];
      }
    
      for (int i = k; i < numsSize; i++) {
        maxSum = max(maxSum, currentSum);
    
        currentSum += nums[i] - nums[i - k];
      }
    
      return maxSum;
    }
    
    int max(int a, int b) {
      if (a > b)
        return a;
      return b;
    }
\end{lstlisting}

\pagebreak
\section{Two Sum}
The two sum algorithm is a simple one used when needing to find if two numbers in a list can be added to equate to a given target.\newline

For the simplest form of two sum (exemplified below), the algorithm goes as such:

\begin{itemize}
    \item Loop through input
    \item Calculate the difference between the current item and the target
    \item If the complement is in the hashmap, return the current index and the index of the complement
    \item Add the current number to a hashmap as a key, who's value is the index
\end{itemize}

This can be done in $O(n)$ time. The reason this works is because at every iteration of our loop, we're seeing if we already came across a number (an appropriate complement of the current number) that will make the current number equal our target when added together. 

\begin{lstlisting}[language=Python, caption={Two Sum in Python}]
    def two_sum(nums, target):
        complements = {}
        for i in range(len(arr)):
            diff = target - nums[i]
    
            if diff in complements:
                return [i, complements[diff]]
            complements[nums[i]] = i

    two_sum([9,6,11,15], 17)
    
\end{lstlisting}

\begin{lstlisting}[language=C++, caption={Two Sum in C++}]
    vector<int> twoSum(vector<int>& nums, int target) {
        map<int, int> complements;
        vector<int> result;

        for(int i = 0; i < nums.size(); i++) {
          if (complements.count(target - nums[i]) > 0) {
            return {i, complements[target - nums[i]]};
          }
          complements.insert(pair<int,int>(nums[i], i));
        }

        return {-1, -1};
    }
\end{lstlisting}

\pagebreak
\section{Top K Frequent Elements}



\end{document}

